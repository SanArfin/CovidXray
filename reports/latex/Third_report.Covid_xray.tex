\documentclass{article}
\usepackage{float}
\usepackage{hyperref}
\usepackage{graphicx} % Required for inserting images
\usepackage{multirow} % required for advanced tables
\usepackage[english]{babel}
\usepackage[T1]{fontenc}
\usepackage{textcomp}
%\usepackage{titlesec}
\usepackage{parskip}
\usepackage[margin=1in]{geometry}
\usepackage{color}
\usepackage{subfig}
%\usepackage{subcaption}

\graphicspath{{../figures/figures_report_3/}}

\title{ANALYSIS OF COVID-19 CHEST X-RAYS: \\Report 3: Final Report}
\author{Saniya Arfin, Yvonne Breitenbach, Alexandru Buzgan}
\date{June 2025}

\begin{document}

\maketitle

\tableofcontents

\newpage 

% ------------------------------------------------------------------------
% Chapter 1: Introduction
%------------------------------------------------------------------------
\section{Introduction}
Some ideas: \\
Background and Project Motivation\\
Describe the context of the project, the public health importance of chest X-ray analysis, and the impetus for the study.\\
Outlines the background, motivation, objectives, scope, and structural layout of the report.\\



%------------------------------------------------------------------------
% Chapter 2: Dataset Overview
%------------------------------------------------------------------------
\section{Dataset Overview - Description of the "Chest-X-Ray" Dataset}
Some ideas: \\
Describes the "Chest-X-Ray" dataset in detail, discussing its composition, class imbalance issues, and data quality challenges, along with the preprocessing techniques applied.\\
source of the data\\


%------------------------------------------------------------------------
% Chapter 3: Methodology - Modelling
%------------------------------------------------------------------------
\section{Modelling}
Some ideas\\
Summary of the modelling. Must be content from our second report, but more filtered. Maybe only the best try from each model. And some graphical overview of how good theses "best" candidates are. So that the reader get's an overview. Don't include all the details!!! \\
Findings from traditional machine learning, findings from deep learning, and transfer learning. 


%------------------------------------------------------------------------
% Chapter 4: Conclusion and Future Work
%------------------------------------------------------------------------
\section{Conclusion and Future Work}

Some ideas:\\
Draw a Conclusion from the previous chapter: \\
Which is the model we chose? Why do we chose it? \\
How can this model be used? What are the limitatios? \\
What has someone to keep in mind, when using the model? 


%------------------------------------------------------------------------
% Chapter 5: Discussion
%------------------------------------------------------------------------
\section{Discussion - Reflecting on our project}

some ideas: \\
Reflection on our work. \\
Interprets the experimental results, identifies strengths and limitations, and discusses potential clinical implications.\\

Questions we could answer, FROM DS: Report methodology:\\

Difficulties encountered during the project: 
\begin{itemize}
    \item What was the main scientific obstacle encountered during this project?
    \item For each of the following points, if you encountered difficulties, detail how they slowed you down in setting up your project.
    \item Forecast: tasks that took longer than expected, etc.
    \item Datasets: acquisition, volumetry, processing, aggregation, etc.
    \item Technical/theoretical skills: timing of skill acquisition, skill not offered in training, etc.
    \item Relevance: of the approach, model, data, etc.
    \item IT: storage power, computational power, etc.
    \item Other
\end{itemize}



%------------------------------------------------------------------------
% Chapter 6: References
%------------------------------------------------------------------------
\section{References}

Lists all the sourced literature and references cited throughout the report.

%------------------------------------------------------------------------
% Chapter 7: Appendices
%------------------------------------------------------------------------
\section{Appendices}

??? \\
Gantt diagram. \\
Description of code files. \\
You can find the GitHub repository belonging to this project here: 
\url{https://github.com/DataScientest-Studio/mar25-bds_analysis-of-covid-19-chest-x-rays} 


\end{document}
